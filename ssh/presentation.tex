\documentclass[10pt,graphics,aspectratio=169,table]{beamer}
\usepackage{listings}
\usepackage{hyperref}
\usetheme{metropolis}

\title{SSH}
\author{Jakob Krebs}
\date{ESE 2019}
\institute{NERD101 - ESE - ifsr - TU Dresden}
\titlegraphic{\hfill\includegraphics[height=1.25cm]{../eselogo}}
\begin{document}
\maketitle
\begin{frame}{Outline}
    \tableofcontents
\end{frame}

% motivation
\begin{frame}{motivation}
you know how powerful your shell ist and want to use it on other computers.
Reasons for servers:
    \begin{itemize}
        \item uptime
        \item location
        \item compute power
        \item connectivity
    \end{itemize}
\end{frame}
% verbinden zum server
\begin{frame}[fragile]{connect to server}
    \begin{lstlisting}[language=bash]
 ssh user@example.com
user@example ~>
    \end{lstlisting}
\end{frame}
% identity files
\section{Keyfiles}
\begin{frame}[fragile]{keyfiles}
    \begin{itemize}
        \item security
        \item laziness
    \end{itemize}
    generate key
    \begin{lstlisting}[language=bash]
 ssh-keygen -t ed25519
    \end{lstlisting}
    copy public key to server
    \begin{lstlisting}[language=bash]
 ssh-copy-id example.com
    \end{lstlisting}
    now you can log in without password
\end{frame}
% proxyjump
\section{ProxyJump}
\begin{frame}[fragile]{Proxyjump}
    balfalsel
\end{frame}
% local und remote portforwarding
\section{port forwardings}
% local
\subsection{local portforwardings}
% remote
\subsection{remote portforwardings}
% config file
\section{config file}
\begin{frame}[fragile]{why use a config file?}
    you don't want to always the username and full hostname
    \begin{lstlisting}
        ssh user@example.com -p 9999 -i my_identity
    \end{lstlisting}
\end{frame}

\begin{frame}[fragile]{config file}
    instead you can write this into your \$HOME/.ssh/config file 
    \begin{lstlisting}
        Host myhost
        Hostname example.com
        User username
        Port 9999
        IdentityFile PathToMyIdentity
    \end{lstlisting}
\end{frame}


% -G 
\begin{frame}[fragile]{which configs get applied}
    now your config got really big and you want to know which options get applied on command
    \begin{columns}
        \begin{column}{0.5\textwidth}
            \begin{lstlisting}[language=bash]
ssh switch-bu24.agdsn.network -G
user jakob
hostname switch-bu24.agdsn.network
port 22
[...]
proxyjump login.agdsn.tu-dresden.de
    \end{lstlisting}
    \end{column}
        \begin{column}{0.5\textwidth}
            \begin{lstlisting}[language=bash]
Host switch-bu24.agdsn.network
User jakob
Host *
IdentitiesOnly yes
ControlMaster auto
ControlPersist 60s
Host login.agdsn.tu-dresden.de
ProxyCommand none
Hostname 141.76.119.134
Host *.agdsn *.agdsn.network
ProxyJump login.agdsn.tu-dresden.de
IdentityFile ~/.ssh/id_ed25519
        \end{lstlisting}
    \end{column}
    \end{columns}
\end{frame}
% socket reuse
\begin{frame}[fragile]
    sometimes you have a lot of connections to the same server (e.g. in Robolab)\\
so we can reuse our socket to speed up our connections
    \begin{lstlisting}[language=bash]
Host *
ControlMaster auto
ControlPath ~/.ssh/sockets/%r@%h-%p
ControlPersist 600
    \end{lstlisting}
%now we create a socket for each server and reuse it with every session
\end{frame}
% sshuttle
\end{document}
