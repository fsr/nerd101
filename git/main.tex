\documentclass[10pt, graphics, aspectratio=169, table]{beamer}
\usepackage{hyperref}
\usepackage{booktabs}
\usepackage[normalem]{ulem}
\usepackage[flushleft]{threeparttable}

% -- Change to this year's color! --
\definecolor{ese}{RGB}{58, 240, 212}
\newcommand{\ra}{$\Rightarrow$\ }

\usetheme{metropolis}
\setbeamercolor{frametitle}{bg=ese}

\title{Git - version control system}
\author{Armin Wolf}
\date{ESE \the\year{}}
\institute{Nerd::101 - ESE - IFSR - TU Dresden}
\titlegraphic{\hfill\includegraphics[height=2cm]{../logo}}

\begin{document}
    \maketitle

    \begin{frame}{Outline}
        \tableofcontents
    \end{frame}


    \section{Introduction}
    \begin{frame}{Why using a version control system?}
        \begin{columns}
            \column{0.55\textwidth}
                \emph{Developing programs is chaotic}
                \begin{itemize}
                    \item multiple features to track
                    \item multiple releases to support
                    \item multiple developers to manage
                    \item need for rolling back buggy code changes
                    \item \ldots
                \end{itemize}
                \textbf{All possible with a version control system}
            \column{0.45\textwidth}
                \includegraphics[width=\textwidth]{img/chaos.png}
        \end{columns}
    \end{frame}

    \begin{frame}{Git}
        \begin{columns}
            \column{0.55\textwidth}
                Enter: \textbf{Git}
                \begin{itemize}
                    \item multiple workflows \ra feature development, bugfix, \ldots
                    \item language-agnostic \ra wide range of use-cases
                    \item distributed \ra easy to collaborate
                    \item \textbf{open source}
                \end{itemize}
            \column{0.45\textwidth}
                \center\includegraphics[scale=0.4]{img/git.png}
                \center\tiny\url{https://xkcd.com/1597}
        \end{columns}
    \end{frame}


    \section{Basic Concepts}
    \begin{frame}{Repository}
        \begin{columns}
            \column{0.55\textwidth}
                \begin{itemize}
                    \item contains all files monitored by Git
                    \item holds the commit histroy (more later)
                    \item can be local or remote
                \end{itemize}
            \column{0.45\textwidth}
                \center\includegraphics[scale=0.4]{img/repo.png}
                \center\tiny\url{https://www.atlassian.com/de/git/tutorials/setting-up-a-repository/git-clone}
        \end{columns}
    \end{frame}

    \begin{frame}{Repository commands}
        \begin{table}
            \centering
            \begin{threeparttable}
                \begin{tabular}{ll}
                    \toprule
                    Command & Function \\
                    \midrule
                    \texttt{git init <name>} & Creates a new repository \\
                    \texttt{git clone <url>} & Creates a working copy of a remote repository \\
                    \texttt{git status} & Shows the current status of the repository\tnote{1} \\
                    \texttt{git add <file>} & Adds the file \texttt{<file>} to the repository\tnote{1}\\
                    \bottomrule
                \end{tabular}
                \begin{tablenotes}
                    \item [1]\emph{should be executed inside the repository}
                \end{tablenotes}
            \end{threeparttable}
        \end{table}
    \end{frame}

    \begin{frame}{Staging area}
        \begin{columns}
            \column{0.55\textwidth}
                \begin{itemize}
                    \item contains all changes destined ("staged") for commit
                    \item allows for breaking up bigger changes into multiple small commits
                    \item simplifies review and management of complex changes
                \end{itemize}
            \column{0.45\textwidth}
                \center\includegraphics[scale=0.18]{img/staging.png}
                \center\tiny\url{https://codetej.in/3-git-three-stage-architecture/}
        \end{columns}
    \end{frame}

    \begin{frame}{Staging area commands}
        \begin{table}
            \centering
            \begin{threeparttable}
                \begin{tabular}{ll}
                    \toprule
                    Command & Function \\
                    \midrule
                    \texttt{git add <file>} & Stages all changes made to file \texttt{<file>}\tnote{1} \\
                    \texttt{git diff HEAD} & Shows all changes to files inside the repository\tnote{1} \\
                    \texttt{git diff HEAD --cached} & Shows all changes staged for commit\tnote{1} \\
                    \texttt{git restore <file>} & Removes all unstaged changes made to file \texttt{<file>}\tnote{1}\\
                    \texttt{git restore --staged <file>} & Unstages all staged changes made to file \texttt{<file>}\tnote{1}\\
                    \bottomrule
                \end{tabular}
                \begin{tablenotes}
                    \item [1]\emph{should be executed inside the repository}
                \end{tablenotes}
            \end{threeparttable}
        \end{table}
    \end{frame}

    \begin{frame}{Commits}
        \begin{columns}
            \column{0.55\textwidth}
                \begin{itemize}
                    \item set of changes, shows up inside commit history
                    \item contains a topic and description
                    \item ideally self-contained and not too large
                    \item contains a hash sum as an unique identifier
                    \item latest commit is aliased with "HEAD"
                    \item can be reverted
                \end{itemize}
            \column{0.45\textwidth}
                \center\includegraphics[scale=0.5]{img/commits.png}
                \center\tiny\url{https://www.w3docs.com/learn-git/git-commit-amend.html}
        \end{columns}
    \end{frame}

    \begin{frame}{Commits commands}
        \begin{table}
            \centering
            \begin{threeparttable}
                \begin{tabular}{ll}
                    \toprule
                    Command & Function \\
                    \midrule
                    \texttt{git commit} & Commits all staged changes\tnote{1} \\
                    \texttt{git commit <file>} & Commits all staged changes of file \texttt{<file>}\tnote{1} \\
                    \texttt{git commit --amend} & Ammends the staged changes to the latest commit\tnote{1} \\
                    \texttt{git revert <hash>} & Reverts the commit refered to with \texttt{<hash>}\tnote{1} \\
                    \texttt{git log} & Shows the commit history\tnote{1} \\
                    \texttt{git show <hash>} & Shows the changes contained in commit \texttt{<hash>}\tnote{1} \\
                    \bottomrule
                \end{tabular}
                \begin{tablenotes}
                    \item [1]\emph{should be executed inside the repository}
                \end{tablenotes}
            \end{threeparttable}
        \end{table}
    \end{frame}

    \begin{frame}{Branches}
        \begin{columns}
            \column{0.55\textwidth}
                \begin{itemize}
                    \item different lines of development
                    \item useful for developing features, bugfixes, \ldots
                    \item main branch is called "master" or "main"
                    \item each one contains its own commit history
                    \item can be united ("merged") with other branches
                \end{itemize}
            \column{0.45\textwidth}
                \center\includegraphics[scale=0.25]{img/branches.png}
                \center\tiny\url{https://medium.com/@natetadesse4991/git-branches-and-merging-overview-17810959c28a}
        \end{columns}
    \end{frame}

    \begin{frame}{Branches commands}
        \begin{table}
            \centering
            \begin{threeparttable}
                \begin{tabular}{ll}
                    \toprule
                    Command & Function \\
                    \midrule
                    \texttt{git switch <branch>} & Change current branch to \texttt{<branch>}\tnote{1} \\
                    \texttt{git switch -c <branch>} & Creates and changes current branch to \texttt{<branch>}\tnote{1} \\
                    \texttt{git branch} & Shows all branches\tnote{1} \\
                    \texttt{git branch -d <branch>} & Deletes branch \texttt{<branch>}\tnote{1} \\
                    \texttt{git merge <branch>} & Merges branch \texttt{<branch>} into current branch\tnote{1} \\
                    \bottomrule
                \end{tabular}
                \begin{tablenotes}
                    \item [1]\emph{should be executed inside the repository}
                \end{tablenotes}
            \end{threeparttable}
        \end{table}
    \end{frame}

    \begin{frame}{Collaborating with others}
        \begin{columns}
            \column{0.55\textwidth}
                \begin{itemize}
                    \item git is designed to allow easy collaboration
                    \item decentralized \ra forking is common
                    \item usually involves a remote repository, which is then cloned by every collaborater
                \end{itemize}
            \column{0.45\textwidth}
                \center\includegraphics[scale=0.3]{img/collaboration.png}
                \center\tiny\url{https://www.atlassian.com/git/tutorials/comparing-workflows}
        \end{columns}
    \end{frame}

    \begin{frame}{Collaboration commands}
        \begin{table}
            \centering
            \begin{threeparttable}
                \begin{tabular}{ll}
                    \toprule
                    Command & Function \\
                    \midrule
                    \texttt{git fetch} & Fetch new changes from remote repository\tnote{1} \\
                    \texttt{git pull} & Merge changes from remote repository\tnote{1} \\
                    \texttt{git push} & Upload current branch to remote repository\tnote{1} \\
                    \texttt{git remote set-url origin <url>} & Sets new remote repository to \texttt{<url>}\tnote{1} \\
                    \bottomrule
                \end{tabular}
                \begin{tablenotes}
                    \item [1]\emph{should be executed inside the repository}
                \end{tablenotes}
            \end{threeparttable}
        \end{table}
    \end{frame}

    \begin{frame}{Pitfalls}
        \begin{columns}
            \column{0.55\textwidth}
                \begin{itemize}
                    \item commits without a good description are difficult to manage
                    \item commits should be self-contained, since they might be reorderd
                    \item pull before pushing new changes, or else you might experience a merge conflict
                    \item multiple conflicting changes to the same file can also result in a merge conflict
                \end{itemize}
            \column{0.45\textwidth}
                \center\includegraphics[scale=0.45]{img/commit_msg.png}
                \center\tiny\url{https://xkcd.com/1296/}
        \end{columns}
    \end{frame}

    \begin{frame}{Merge conflict resolution}
        \begin{itemize}
            \item git will stop when it encounters a merge conflict
            \item all conflicting changes are market with "==="
            \item choose which version to keep, or combine the two versions
            \item add the resolved file with \texttt{git add <file>}
            \item if all files are clean, continue with \texttt{git merge --continue}
        \end{itemize}
    \end{frame}

    \section{What to do next}
    \begin{frame}{What to do next}
        \begin{columns}
            \column{0.55\textwidth}
                \begin{itemize}
                    \item download git at \url{https://git-scm.com/downloads}
                    \item read the git documentation at \url{https://git-scm.com/docs}
                    \item create an account on github, gitlab, \ldots
                    \item clone a public repository and experiment or participate
                    \item explore more advanced commands
                \end{itemize}
            \column{0.45\textwidth}
                \center\includegraphics[scale=0.28]{img/hosting.png}
                \center\tiny\url{https://codeburst.io/git-from-zero-to-hero-basics-9d8705f3b7dd}
        \end{columns}
    \end{frame}

\end{document}
